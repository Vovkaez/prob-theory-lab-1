\documentclass{article}
\usepackage{graphicx} % Required for inserting images
\usepackage[utf8]{inputenc}
\usepackage[english,russian]{babel}
\usepackage[left=2cm,right=2cm,top=2cm,bottom=2cm,bindingoffset=0cm]{geometry}
\usepackage{mathabx}

\title{Лабораторная работа №1}
\author{Владимир Медведев, группа M3238}
\date{}

\begin{document}

\maketitle
\section*{Задача №2, вариант 1}
Под выбором случайной точки естественно подразумевать равномерное распределение вероятности. Для равномерного распределения вероятности значение вероятности должно быть пропорционально мере множества. Тогда обозначим \\
$ l(x) = \int_{0}^{x} \sqrt{1 + (x^2)'^2}\,dx = \int_{0}^{x} \sqrt{1 + 4x^2} \,dx$ - длина части дуги параболы, начинающейся в точке $\left(0, 0\right)$ и заканчивающейся в точке $\left(x, x^2\right)$. Тогда если зададим вероятность соответствующей дуги равной $P(x) = \frac{l(x)}{l(2)}$, такое распределение будет равномерным. Чтобы вычислить требуемую в задании вероятность определим подмножество данной параболы, где угол на превосходит $\pi / 3$. Очевидно, что оно определяется неравенством
\[\arctan{(x^2)'} \le \pi / 3 \iff \arctan{2x} \le \pi / 3 \ \iff 0 \le 2x \le \sqrt{3} \iff 0 \le x \le \frac{\sqrt{3}}{2}\]
Его вероятность равна $\frac{l(\frac{\sqrt{3}}{2})}{l(2)} = \frac{\ln{(2 + \sqrt{3})} + 2\sqrt{3}}{4l(2)}$. Длина всей параболы равна $l(2) = \frac{\ln{(\sqrt{17} + 4)} + 4\sqrt{17}}{4}$. Тогда ответом на задачу будет являться $\frac{\ln{(2 + \sqrt{3})} + 2\sqrt{3}}{\ln{(\sqrt{17} + 4)} + 4\sqrt{17}} \approx 0.257$.

\section*{Задача №3, вариант 3}
\[P(X = i|X + Y = j) = \frac{P(X = i \bigcap X + Y = j)}{P(X + Y = j)} = \frac{P(X = i \bigcap Y = j - i)}{P(X + Y = j)} = \frac{P(A_i) P(B_{j - i})}{P(X + Y = j)} = \frac{(1 - p) ^ j p ^ 2}{\sum_{x = 0}^{j}{(1 - p) ^ j p ^ 2}} = \frac{1}{j + 1}\]

\section*{Задача №4}

Данные получены с помощью кода, представленного в репозитории. \\
Для значений с $p = 0.5$ вероятность приближается с помощью интегральной теоремы Муавра-Лапласа, так как в данном случае $\frac{n}{2} = np$. В остальных же случаях это приближение работает плохо и вместо него используется локальная теорема Муавра-Лапласа. При $p = 0.5$ приближение, ожидаемо, работает хорошо. Известно, что оно работает лучше, когда $|\frac{1}{2} - p|$ минимален. При $p \not \eq 0.5$ абсолютная погрешность везде достаточно мала.\\\\
Пометка * - вычислить точно не удаётся, данные получены из эксперимента.

\begin{table}
        \begin{tabular}{lllllll}\toprule
            && 0.001 & 0.01 & 0.1 & 0.25 & 0.5 \\\midrule
            10 & Exact & 2.5e-13 & 2.4e-8 & 1.5e-3 & 0.22 & 0.65625 \\
               & Approximate & 0 & 2.75e-53 & 5.8e-5 & 0.226 & 0.6827 \\
            100 & Exact & 9.6e-122 & 6.1e-73 & 3.6-21 & 4.25e-6 & 0.7287 \\
                & Approximate & 0 & 0 & 1.26e-34 & 1.04e-6 & 0.6827 \\
            1000 & Exact & 0 & 0 & 0 & 1.5e-58 & 0.673 \\
                & Approximate & 0 & 0 & 0 & 3.61e-67 & 0.6827 \\
            10000 & Exact & 0* & 0* & 0* & 0* & 0.683* \\
                & Approximate & 0 & 0 & 0 & 0 & 0.6827 \\
            \\\bottomrule
        \end{tabular}
\end{table}
\end{document}
